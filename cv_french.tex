%% LyX 2.0.0 created this file.  For more info, see http://www.lyx.org/.
%% Do not edit unless you really know what you are doing.
\documentclass[12pt,french]{moderncv}
\usepackage[T1]{fontenc}
\usepackage[latin9]{inputenc}
\setcounter{secnumdepth}{2}
\setcounter{tocdepth}{2}
\setlength{\parskip}{\medskipamount}
\usepackage{svg}
\setlength{\parindent}{0pt}
\usepackage[top=0.9in, bottom=0.9in, left=0.9in, right=0.9in]{geometry}

\makeatletter

%%%%%%%%%%%%%%%%%%%%%%%%%%%%%% LyX specific LaTeX commands.
\newcommand{\noun}[1]{\textsc{#1}}

%%%%%%%%%%%%%%%%%%%%%%%%%%%%%% User specified LaTeX commands.
\moderncvtheme[orange]{classic}
% possible themes are "classic" and "casual"
% optional argument are 'blue' (default), 'orange', 'red', 'green', 'grey' and 'roman' (for roman fonts, instead of sans serif fonts)
\usepackage{graphics}
%\usepackage[utf8]{inputenc}
\usepackage[T1]{fontenc}

% required
\vspace{-0.5cm}
\firstname{Antoine}
% required
\familyname{Liutkus}
\address{24 rue du couvent}{34560 Montbazin, France}
\mobile{+33662227080}
\extrainfo{n\'e en 1981, Francais}


% optional, remove the line if not wanted
\title{Charg\'e de recherche, Inria }

% optional
\email{antoine.liutkus@inria.fr}

% optional
% \photo[height]{name}
% 'height' is the height the picture is resized to
% 'name' is the name of the picture file
\photo[80pt]{antoine}

% optional
\quote{}

\makeatother

\usepackage{babel}
\addto\extrasfrench{%
   \providecommand{\og}{\leavevmode\flqq~}
   \providecommand{\fg}{\ifdim\lastskip>\z@\unskip\fi~\frqq}
}

\begin{document}
\maketitle
\vspace{-2.5cm}
\section{Activit\'es de recherche}
\cvitem{Th\'eorie}
{Apprentissage automatique, traitement du signal, probabilit\'es\begin{itemize}
\item[$\bullet~$] \emph{Apprentissage profond}
\item[$\bullet~$] \emph{Processus stochastiques et inf\'erence Bay\'esienne}
\item[$\bullet~$] \emph{Transport optimal}
\item[$\bullet~$] \emph{M\'ethodes alg\'ebriques: factorisations tensorielles}
\end{itemize}}
\vspace{-0.5cm}
\cvitem{Applications}
{Apprentissage automatique pour l'audio\begin{itemize}
\item[$\bullet~$] \emph{S\'eparation de sources et d\'ebruitage}
\item[$\bullet~$] \emph{Filtrage multicanal robuste}
\item[$\bullet~$] \emph{Localisation de sources}
\item[$\bullet~$] \emph{\'echantillonage compress\'e}
\end{itemize}}
\vspace{-0.5cm}
\cvitem{Publications}
{\href{https://scholar.google.fr/citations?hl=fr&user=0s7V4FAAAAAJ}{\includegraphics[width=15pt]{linksolid.png}}}
\vspace{-0.3cm}
\section{Positions}
\cventry{2014--today}{Researcher}{Inria, France}{}{}{S\'eparation de sources, deep learning, supervision et coordination scientifique}
\cventry{2013}{Postdoc}{Institut Langevin, ESPCI, Paris, France}{}{}{\'echantillonage compress\'e, parcimonie, optique non conventionelle}
\cventry{2010--2012}{Ph.D}{Telecom Paristech, CNRS LTCI, Paris, France}{}{}{Codage audio et s\'eparation}
\cventry{2010--2012}{Enseignant en traitement du signal}{Engineering schools \textit{ESME Sudria, ISEP, Telecom ParisTech}, Paris, France}{}{}{}
\cventry{2007--2010}{Ing\'enieur de recherche}{Audionamix, Paris, France}{}{}{D\'emixage musical et revalorisation d'enregistrements historiques}


\section{Cursus}

\cventry{2010--2012}{Ph.D}{S\'eparation de sources inform\'ee}{Supervis\'e by Ga\"el Richard et Roland Badeau}{Telecom ParisTech, France}{}

\cventry{2004--2005}{Master 2}{Acoustique \& Traitement du signal}{IRCAM, Paris}{}{}

\cventry{1999--2004}{Ing\'enieur}{Telecom ParisTech, Paris, France}{}{}{}

\section{Service scientifique}

\cvitem{Comit\'es internationaux}
{\vspace{-0.4cm}\begin{itemize}
\item[$\bullet~$]  IEEE Tech. Committee on Audio \& Acoustics (2016-2021)
\item[$\bullet~$]  Separation Eval. Campaign (SiSEC) general chair (2016-2018)
\end{itemize}}
\cvitem{Relectures}
{\vspace{-0.4cm}\begin{itemize}
\item[$\bullet~$] journaux : \emph{IEEE SPL, TASLP, TSP, SPM, Elsevier SP, DSP}
\item[$\bullet~$] conf\'erences: \emph{NeurIPS, ICASSP, EUSIPCO, ISMIR, WASPAA, ...}
\end{itemize}}
\vspace{-0.7cm}


\end{document}
